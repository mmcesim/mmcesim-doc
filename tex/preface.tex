\begin{tip}
  mmCEsim documentation \& tutorials are under development!
\end{tip}

As a researcher in wireless communications and signal processing,
I have always had a passion for programming.
Since my first year at university
when I started using C++ to accomplish amazing tasks,
I have been convinced of the importance of software in research.

Despite this, many researchers underestimate the significance of software
in implementation, simulation, and verification of algorithms.
Scientific software and programming languages, along with libraries,
have been the driving force behind advances in science.
Therefore, I am proud to present mmCEsim,
an open-source software that is not only easy to use but also free for all.

The idea for mmCEsim originated from the tedious process of writing C++ code
for implementing compressed channel estimation for
reconfigurable intelligent surface (RIS)-assisted multiple-input multiple-output (MIMO) systems.
I was driven by a desire to get rid of these repetitive tasks and eliminate the need
to spend so much time setting up simulations.
Inspired by NYUSIM, I decided to create my own simulation software.

To make it even easier to use, I designed a programming language called ALG,
with simple syntax for describing algorithms.
This language can be converted into other languages, such as C++ and \textsc{Matlab}, for simulation.
To use mmCEsim, simply configure your system settings, decide on your channel estimation algorithm,
and extend the sounding and estimation process with ALG.

At present, mmCEsim supports channel estimation based on compressed sensing in mmWave
and is expected to be more general in the future.
It is still under active development and evolving.

I would like to thank my professor, seniors, and fellow students for their help and inspiration.
I would also like to express my gratitude to Jinwen Xu for designing the elegant \LaTeX{} template,
\texttt{beaulivre}, which has made this document possible.

\begin{flushright}
  \textsc{Wuqiong Zhao}

  San Diego, CA, U.S.A.

  \TheDate{\the\year/\the\month}[only-year-month]
\end{flushright}
