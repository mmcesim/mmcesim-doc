\section{Publications}

A brief introduction of mmCEsim is given in the
\href{https://pub.mmcesim.org/mmCEsim_Nanjing2022_Poster.pdf}{poster}
at the 2022 National Postdoc Seminar in Nanjing,
which I attend as the only undergraduate student,
and got the Honorable Mention award.

This document is also published online at \url{https://pub.mmcesim.org/mmCEsim-doc.pdf}.

\section{Websites}

\subsection{For Users}

If you are the user of mmCEsim and wants to know more,
you may find the following websites in Table~\ref{a:tab:web_user} useful.
\begin{table}[htbp]
  \caption{Websites for users.}
  \label{a:tab:web_user}
  \renewcommand{\arraystretch}{1.2}
  \begin{tabularx}{\linewidth}{cX}
    \toprule
    \tbhead{Website} & \tbhead{URL} \\
    \midrule
    Homepage & \url{https://mmcesim.org} \\
    Web Application & \url{https://app.mmcesim.org} \\
    Blog & \url{https://blog.mmcesim.org} \\
    Publications & \url{https://pub.mmcesim.org} \\
    VS Code Extension & \url{https://marketplace.visualstudio.com/items?itemName=mmcesim.mmcesim} \\
    \bottomrule
  \end{tabularx}
\end{table}

\subsection{For Developers}
If you are a developer and maybe want to contribute to the mmCEsim project,
you can find additional websites in Table~\ref{a:tab:web_dev}.
\begin{table}[htbp]
  \caption{Websites for developers.}
  \label{a:tab:web_dev}
  \renewcommand{\arraystretch}{1.2}
  \begin{tabularx}{\linewidth}{cX}
    \toprule
    \tbhead{Website} & \tbhead{URL} \\
    \midrule
    GitHub Organization & \url{https://github.com/mmcesim} \\
    C++ Dev Documentation & \url{https://dev.mmcesim.org} \\
    CLI App Wiki & \url{https://github.com/mmcesim/mmcesim/wiki} \\
    \bottomrule
  \end{tabularx}
\end{table}

\section{Author}
\textbf{Wuqiong Zhao} (\textit{Student Member, IEEE})
is an undergraduate student pursuing the Bachelor's Degree in communications engineering, working at Lab of Efficient Architectures for Digital-communication and Signal-processing (LEADS) and National Mobile Communications Research Laboratory, Southeast University.
He is the honors student of Chien-Shiung Wu College
and earned the National Scholarship and Cyrus Tang Scholarship in 2021.
From 2020 to 2021, he also served as the Special Student Assistant to President of Southeast University.
He was also nominated as the most influential undergraduate student of Southeast University in 2022.
His research interest includes channel estimation, Bayesian algorithms, and the intelligent reflecting surface (IRS) in wireless communication of 5G and 6G.
% He assisted editing the book \textit{Channel Codes for 5G Wireless Systems} and the chapter \textit{Stochastic Computation for Baseband Processing}.
He is also the reviewer of IEEE TCAS II and ISCAS 2023.
