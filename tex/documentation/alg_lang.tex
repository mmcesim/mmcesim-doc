\section{Data Type}

\subsection{Why Need Data Type}
Languages Python and Matlab/Octave are weakly typed
which can be convenient for writing the code.
However, that is problematic for implementation.
The efficiency is not satisfactory compared to C++,
and sometimes you may encounter ambiguous error information in Matlab.
Therefore, for the sake of efficiency and generality,
ALG language is designed to be \textbf{strongly typed}.

\subsection{Structure}
The type specification is very simple,
because ALG language concentrates on matrices.
Basically, the structure of ALG language is
\[
  \text{prefix}+\text{dimension}+\text{suffix}.
\]
For example, \texttt{f2c} means a matrix (dimension is 2) with data type as float
and property as a constant.

\subsection{Specifiers}

\subsubsection{Prefix}

\paragraph{Basic Type Prefix}

Basic type just names the element type.
They are shown in Table~\ref{d:tab:basic_type_prefix}.
\begin{table}[htbp]
  \caption{Basic type prefix.}
  \label{d:tab:basic_type_prefix}
  \renewcommand{\arraystretch}{1.2}
  \begin{tabularx}{\linewidth}{ccYYY}
    \toprule
    \tbhead{Predix} & \tbhead{Type} & \tbhead{C++ Type} & \tbhead{Python Type} & \tbhead{Matlab/Octave Type} \\
    \midrule
    \texttt{c} & Complex &
    \href{https://arma.sourceforge.net/docs.html\#cx_double}{\texttt{cx\_double}}
    & \texttt{complex} &
    \href{https://www.mathworks.com/help/matlab/ref/complex.html}{\texttt{complex}} \\
    \texttt{f} & Float & \texttt{double} & \texttt{double} &
    \href{https://www.mathworks.com/help/matlab/ref/double.html}{\texttt{double}} \\
    \texttt{i} & Integer & \texttt{int} & \texttt{int} &
    \href{https://www.mathworks.com/help/matlab/ref/int64.html}{\texttt{int64}} \\
    \texttt{u} & Unsigned Integer &
    \href{https://arma.sourceforge.net/docs.html\#uword}{\texttt{uword}} &
    \texttt{uint} &
    \href{https://www.mathworks.com/help/matlab/ref/uint64.html}{\texttt{uint64}} \\
    \texttt{b} & Boolean & \texttt{bool} & \texttt{bool} &
    \href{https://www.mathworks.com/help/matlab/ref/logical.html}{\texttt{logical}} \\
    \texttt{s} & String &
    \href{https://en.cppreference.com/w/cpp/string/basic_string}{\texttt{std::string}}
    & \texttt{str} &
    \href{https://www.mathworks.com/help/matlab/ref/string.html}{\texttt{string}} \\
    \texttt{h} & Character & \texttt{char} & \texttt{char} &
    \href{https://www.mathworks.com/help/matlab/ref/char.html}{\texttt{char}} \\
    \bottomrule
  \end{tabularx}
\end{table}

\paragraph{Alias Prefix}
Alias prefixes not only set the element type,
but also the dimension.
They are the one character alias for a two-character type.

\section{Function}

\section{Calculation (CALC)}

\section{Macro}

\section{ALG Library}
