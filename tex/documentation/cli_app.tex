\section{CLI Options}\index{CLI options}

\subsection{Help Yourself}\label{d:subsec:cli_opt_help_yourself}

With \texttt{mmcesim -h}, you can view all supported commands and options.
\begin{lstlisting}
mmCEsim 0.2.1  (C) 2022-2023 Wuqiong Zhao
Millimeter Wave Channel Estimation Simulation
=============================================

Usage: mmcesim <command> <input> [options]

Commands:
  sim [ simulate ]       run simulation
  dbg [ debug ]          debug simulation settings
  exp [ export ]         export code
  cfg [ config ]         configure mmCEsim options
  (Leave empty)          generic use

Allowed options:

Generic options:
  -v [ --version ]       print version string
  -h [ --help ]          produce help message
  --gui                  open the GUI app

Configuration:
  -o [ --output ] arg    output file name
  -s [ --style ] arg     style options (C++ only, with astyle)
  -l [ --lang ] arg      export language or simulation backend
  --value arg            value for configuration option
  -f [ --force ]         force writing mode
  -V [ --verbose ]       print additional information
  --no-error-compile     do not raise error if simulation compiling fails
  --no-term-color        disable colorful terminal contents
\end{lstlisting}

\subsection{Command}\index{CLI command}

The allowed commands are explained in the following.

\subsubsection{exp}\indexCLIcmd{exp}\indexCLIcmd{export}
Command \texttt{exp} exports the \texttt{.sim} configuration and corresponding \texttt{.alg} algorithms
to a selected language.
Currently, only export to C++ with Armadillo is supported.

\subsubsection{sim}\indexCLIcmd{sim}\indexCLIcmd{simulate}\label{d:subsec:sim}
Command sim simulates the exported code with the selected backend. Currently, only C++ with Armadillo is supported.

So far, only C++ compiler \texttt{g++}\indextt{g++} (default) and \texttt{clang++}\indextt{clang++} are supported which can be configured with option \hyperref[d:subsec:cfg]{\texttt{cfg}} \texttt{cpp}\indextt{cpp}.
You may also need to configure additional C++ flags with \texttt{cfg cppflags}\indextt{cppflags} if by default the compiler cannot find
\href{https://arma.sourceforge.net/}{armadillo}
library.

\subsubsection{dbg}\indexCLIcmd{dbg}\indexCLIcmd{debug}
Debug the simulation.
This is different from \hyperref[d:subsec:sim]{\texttt{sim}} in that the generated C++ code is compiled with \texttt{-g3} instead of \texttt{-O3}.
Therefore, debug information is retained.

\subsubsection{cfg}\indexCLIcmd{cfg}\indexCLIcmd{config}\label{d:subsec:cfg}
Configure settings.

\begin{itemize}
  \item Use \texttt{mmcesim cfg \textit{<name>}} to show the value of \texttt{\textit{<name>}}.
  \item Use \texttt{mmcesim cfg \textit{<name>} --value=\texttt{<name>}} to set the value of \texttt{\textit{<name>}} as \texttt{\textit{<value>}}.
\end{itemize}

\begin{example}[Configure C++]~
  \begin{lstlisting}[language=sh]
mmcesim cfg cpp --value="clang++"
mmcesim cfg cppflags --value="-I/opt/local/include -L/opt/local/lib"
  \end{lstlisting}
  {\small Source: \url{https://github.com/mmcesim/mmcesim/blob/master/scripts/mac_config_cppflags_tvj.sh}.}
\end{example}

\subsection{Options}\index{CLI option}

\subsubsection{\texttt{-v} (\texttt{-{}-version})}\indexCLIopt{-v}\indexCLIopt{-{}-version}
Print the version string of mmCEsim.

\subsubsection{\texttt{-h} (\texttt{-{}-help})}\indexCLIopt{-h}\indexCLIopt{-{}-help}
See \S\ref{d:subsec:cli_opt_help_yourself}.

\section{Configuration}\index{configuration}

\subsection{\texttt{version}}\indextt{version}
This field takes a string value representing the targeted mmCEsim version.
For compatibility convenience, this string can be used by the compiler
to decide the behavior.
The current default value is the same as the compiler version (\texttt{\mmCEsimVersion}).

\subsection{\texttt{meta}}\indextt{meta}
This is a map that provides metadata which can be used in the report.
The used fields now include
\texttt{title}\indexTtt{meta}{title},
\texttt{description}\indexTtt{meta}{description},
\texttt{author}\indexTtt{meta}{author}.

\subsection{\texttt{physics}}\indextt{physics}
This field is a map that contains physical system settings.

\subsubsection{\texttt{frequency}}\indextt{frequency}.


\section{Algorithm}

\section{Tools}

\subsection{Compose}\index{compose}\indextt{mmcesim-compose}

\subsection{Log}\index{log}\indextt{mmcesim-log}

\subsection{Maintain}\index{maintain}\indextt{mmcesim-maintain}
