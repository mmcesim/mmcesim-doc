\section{CLI Options}\index{CLI options}

\subsection{Help Yourself}\label{d:subsec:cli_opt_help_yourself}

With \texttt{mmcesim -h}, you can view all supported commands and options.
\begin{lstlisting}
mmCEsim 0.2.1  (C) 2022-2023 Wuqiong Zhao
Millimeter Wave Channel Estimation Simulation
=============================================

Usage: mmcesim <command> <input> [options]

Commands:
  sim [ simulate ]       run simulation
  dbg [ debug ]          debug simulation settings
  exp [ export ]         export code
  cfg [ config ]         configure mmCEsim options
  (Leave empty)          generic use

Allowed options:

Generic options:
  -v [ --version ]       print version string
  -h [ --help ]          produce help message
  --gui                  open the GUI app

Configuration:
  -o [ --output ] arg    output file name
  -s [ --style ] arg     style options (C++ only, with astyle)
  -l [ --lang ] arg      export language or simulation backend
  --value arg            value for configuration option
  -f [ --force ]         force writing mode
  -V [ --verbose ]       print additional information
  --no-error-compile     do not raise error if simulation compiling fails
  --no-term-color        disable colorful terminal contents
\end{lstlisting}

\subsection{Command}\index{CLI command}

The allowed commands are explained in the following.

\subsubsection{exp}\indexCLIcmd{exp}\indexCLIcmd{export}
Command \texttt{exp} exports the \texttt{.sim} configuration and corresponding \texttt{.alg} algorithms
to a selected language.
Currently, only export to C++ with Armadillo is supported.

\subsubsection{sim}\indexCLIcmd{sim}\indexCLIcmd{simulate}\label{d:subsec:sim}
Command sim simulates the exported code with the selected backend. Currently, only C++ with Armadillo is supported.

So far, only C++ compiler \texttt{g++}\indextt{g++} (default) and \texttt{clang++}\indextt{clang++} are supported which can be configured with option \hyperref[d:subsec:cfg]{\texttt{cfg}} \texttt{cpp}\indextt{cpp}.
You may also need to configure additional C++ flags with \texttt{cfg cppflags}\indextt{cppflags} if by default the compiler cannot find
\href{https://arma.sourceforge.net/}{armadillo}
library.

\subsubsection{dbg}\indexCLIcmd{dbg}\indexCLIcmd{debug}\label{d:subsec:dbg}
Debug the simulation.
This is different from \hyperref[d:subsec:sim]{\texttt{sim}} in that the generated C++ code is compiled with \texttt{-g3} instead of \texttt{-O3}.
Therefore, debug information is retained.

\subsubsection{cfg}\indexCLIcmd{cfg}\indexCLIcmd{config}\label{d:subsec:cfg}
Configure settings.

\begin{itemize}
  \item Use \texttt{mmcesim cfg \textit{<name>}} to show the value of \texttt{\textit{<name>}}.
  \item Use \texttt{mmcesim cfg \textit{<name>} --value=\texttt{<name>}} to set the value of \texttt{\textit{<name>}} as \texttt{\textit{<value>}}.
\end{itemize}

\begin{example}[Configure C++]~
  \begin{lstlisting}[language=sh]
mmcesim cfg cpp --value="clang++"
mmcesim cfg cppflags --value="-I/opt/local/include -L/opt/local/lib"
  \end{lstlisting}
  {\small Source: \url{https://github.com/mmcesim/mmcesim/blob/master/scripts/mac_config_cppflags_tvj.sh}.}
\end{example}

\subsection{Options}\index{CLI option}

\subsubsection{\texttt{-v} (\texttt{-{}-version})}\indexCLIopt{-v}\indexCLIopt{-{}-version}
Print the version string of mmCEsim.

\subsubsection{\texttt{-h} (\texttt{-{}-help})}\indexCLIopt{-h}\indexCLIopt{-{}-help}
See \S\ref{d:subsec:cli_opt_help_yourself}.

\subsubsection{\texttt{-{}-gui}}\indexCLIopt{-{}-gui}
Open the GUI application.

\subsubsection{\texttt{-o} (\texttt{-{}-output})}\indexCLIopt{-o}\indexCLIopt{-{}-output}
Set the output file name.
No extension name is required, and is added automatically according to your backend settings.
\texttt{.cpp} for C++, \texttt{.py} for Python, \texttt{.ipynb} for Jupyter,
and \texttt{.m} for \textsc{Matlab} or GNU Octave.

\subsubsection{\texttt{-s} (\texttt{-{}-style})}\indexCLIopt{-s}\indexCLIopt{-{}-style}
Set C++ \href{https://astyle.sourceforge.net/astyle.html}{AStyle} (code formatting) options.

\subsubsection{\texttt{-l} (\texttt{-{}-lang})}\indexCLIopt{-l}\indexCLIopt{-{}-lang}
Set the export language or simulation backend.

\subsubsection{\texttt{-{}-value}}\indexCLIopt{-{}-value}
The value for configuration options.

\subsubsection{\texttt{-f} (\texttt{-{}-force})}\indexCLIopt{-f}\indexCLIopt{-{}-force}
Enable the force writing mode.
This will overwrite existent output files.

\subsubsection{\texttt{-V} (\texttt{-{}-verbose})}\indexCLIopt{-V}\indexCLIopt{-{}-verbose}
Print additional information.

\subsubsection{\texttt{-{}-no-error-compile}}\indexCLIopt{-{}-no-error-compile}
Do not raise error if compiling fails.
This is useful in \hyperref[d:subsec:sim]{\texttt{sim}} and \hyperref[d:subsec:dbg]{\texttt{dbg}}.

\subsubsection{\texttt{-{}-no-term-color}}\indexCLIopt{-{}-no-term-color}
Disable colorful terminal contents.

\begin{tip}
  mmCEsim also supports the \href{https://no-color.org}{\texttt{NO\_COLOR}}\indextt{NO\_COLOR} standard:
  \textit{Command-line software which adds ANSI color to its output by default
    should check for a \texttt{NO\_COLOR} environment variable that,
    when present and not an empty string (regardless of its value),
    prevents the addition of ANSI color.}
\end{tip}

When you have a non-empty \texttt{NO\_COLOR}\indextt{NO\_COLOR} environmental variable,
the color output is disabled,
and you no longer need the \texttt{-{}-no-term-color}\indextt{-{}-no-term-color} option.

\section{Configuration}\index{configuration}

Configuration is written in a text file with extension \texttt{.sim} (actually, can be any file extension)
in YAML\index{YAML} syntax.
\textcolor{Red}{\textsf{[Required]}} keys need to be filled in, unless provided with a
\textcolor{forestgreen}{\textsf{Default}} value.
\textcolor{Orange}{\textsf{[Optional]}} keys can be specified.
\textcolor{Purple}{\textsf{[Conditional]}} keys can be either required or optional,
depending on other settings.

\ExplSyntaxOn

% #2: name
% #3: required (1) or optional (0, default)
% #4: default value (only when required)
\NewDocumentCommand { \CLIConfigLevelI } { o m O{0} o }
  {
    \subsection[#2]
      {
        \texttt{#2}
        \IndexDotfill
        \tl_if_eq:nnTF { #3 } { 1 }
          {
            \IfNoValueTF{#4}{}{\textcolor{forestgreen}{Default:~}\textcolor{black}{\texttt{#4}}\quad}
            \textcolor{Red}{[Required]}
          }
          {
            \tl_if_eq:nnTF { #4 } { 2 }
            {
              \IfNoValueTF{#4}{}{\textcolor{forestgreen}{Default:~}\textcolor{black}{\texttt{#4}}\quad}
              \textcolor{Purple}{[Conditional]}
            }
            {
              \textcolor{Orange}{[Optional]}
            }
          }
      }
    \indextt{#2}
    \IfNoValueTF{#1}
      {
        \label{d:subsec:sim_config_#2}
      }
      {
        \label{d:subsec:sim_config_#1}
      }
  }

% #3: Level I
% #4: name
% #5: required (1) or optional (0, default)
% #6: default value (only when required)
\NewDocumentCommand { \CLIConfigLevelII } { o o m m O{0} o }
  {
    \subsubsection[#4]
      {
        \textcolor{gray}{\texttt{#3}\,/\,}\textcolor{black}{\texttt{#4}}
        \IndexDotfill
        \tl_if_eq:nnTF { #5 } { 1 }
          {
            \IfNoValueTF{#6}{}{\textcolor{forestgreen}{Default:~}\textcolor{black}{\texttt{#6}}\quad}
            \textcolor{Red}{[Required]}
          }
          {
            \tl_if_eq:nnTF { #5 } { 2 }
            {
              \IfNoValueTF{#6}{}{\textcolor{forestgreen}{Default:~}\textcolor{black}{\texttt{#6}}\quad}
              \textcolor{Purple}{[Conditional]}
            }
            {
              \textcolor{Orange}{[Optional]}
            }
          }
      }
    \indexTtt{#3}{#4}
    \IfNoValueTF{#1}
      {
        \label{d:subsubsec:sim_config_#3_#4}
      }
      {
        \label{d:subsubsec:sim_config_#1_#2}
      }
  }

\NewDocumentCommand { \CLIConfigLevelIII } { o o o m m m O{0} o }
  {
    \vspace*{-.8\baselineskip}
    \paragraph[#6]
      {
        \textcolor{gray}{$\blacktriangleright$\quad\texttt{#4}\,/\,\texttt{#5}\,/\,}\textcolor{black}{\texttt{#6}}
        \IndexDotfill
        \tl_if_eq:nnTF { #7 } { 1 }
          {
            \IfNoValueTF{#8}{}{\textcolor{forestgreen}{Default:~}\textcolor{black}{\texttt{#8}}\quad}
            \textcolor{Red}{[Required]}
          }
          {
            \tl_if_eq:nnTF { #7 } { 2 }
            {
              \IfNoValueTF{#8}{}{\textcolor{forestgreen}{Default:~}\textcolor{black}{\texttt{#8}}\quad}
              \textcolor{Purple}{[Conditional]}
            }
            {
              \textcolor{Orange}{[Optional]}
            }
          }
      }
    \indexTTtt{#4}{#5}{#6}
    \IfNoValueTF{#1}
      {
        \label{d:par:sim_config_#4_#5_#6}
      }
      {
        \label{d:par:sim_config_#1_#2_#3}
      }
    \mbox{\hspace*{-1em}}
    \par
  }
\newcommand{\CLIConfigEmpty}{\textnormal{\textit{<empty>}}}
\ExplSyntaxOff

\CLIConfigLevelI{version}[1][\mmCEsimVersion]
This field takes a string value representing the targeted mmCEsim version.
For compatibility convenience, this string can be used by the compiler
to decide the behavior.
The current default value is the same as the compiler version (\texttt{\mmCEsimVersion}).

\CLIConfigLevelI{meta}
This is a map that provides metadata which can be used in the report.
The used fields now include
\texttt{title}\indexTtt{meta}{title},
\texttt{description}\indexTtt{meta}{description},
\texttt{author}\indexTtt{meta}{author}.

\CLIConfigLevelI{physics}
This field is a map that contains physical system settings.

\CLIConfigLevelII{physics}{frequency}[1][narrow]
The frequency bandwidth is specified in this field,
which can have value \texttt{narrow} for narrowband\index{narrowband} (default)
or \texttt{wide} for wideband\index{wideband}.

\CLIConfigLevelII[physics][off_grid]{physics}{off\_grid}[1][true]
This is actually about the model.
With the geometric channel model with grid,
there can be off-grid (or power leakage) problems.
Recently, there are also super resolution formulations to solve the problem.
But we still adopt the grid representation for its popularity and simplicity.
By setting \texttt{off\_grid} to false,
the off grid effect is discarded, i.e.\@ all angles fall on the grid.
The default value is \texttt{true}.

\CLIConfigLevelII{physics}{carriers}[2]
For a wideband system,
you may specify the number of carriers used in OFDM\index{OFDM}.
Its corresponding
macro % TODO
in
\texttt{CALC} % TODO
is \ALG{`CARRIERS_NUM`}.

\CLIConfigLevelI{nodes}[1]
A sequence (array) of nodes in the channel network.
Transmitter (Tx), Receiver (Rx), Reconfigurable Intelligent Surface (RIS)
are all considered node (channels are the connecting edges to these nodes).
For each of its elements, you need to specify the following fields.

\CLIConfigLevelII{nodes}{id}[1]
The \texttt{id} is used in \hyperref[d:subsec:sim_config_channels]{\texttt{channels}} so that we know the direction of channel.

\CLIConfigLevelII{nodes}{role}[1][RIS]
Valid values include
\texttt{transmitter} (\texttt{Tx}),
\texttt{receiver} (\texttt{Rx}),
and \texttt{RIS} (\texttt{IRS}, default value).

There should be one and only one transmitter and one receiver.

\CLIConfigLevelII{nodes}{num}[1][1]
The field \texttt{num} can be used to replicate several copies of the same node.
This is often used for multi-user scenarios.
Currently, this value is discarded, and the number is always 1.

\CLIConfigLevelII{nodes}{size}[1]
The size means the antenna/element number for a node.
For transmitters and receivers, it is the antenna number.
For RIS, it is the number of reflecting elements.
The value is a scalar for uniform linear array (ULA),
and is a 2-value array (for example \texttt{[8, 4]}) for uniform planar array (UPA).
For a 2-value array that has the second value set to 1,
it is still regarded as a ULA.

Its corresponding \hyperref[d:sec:macro]{macro}s in \hyperref[d:sec:calc]{CALC} are
\ALG{`SIZE.T.x`}, \ALG{`SIZE.T.y`}, \ALG{`SIZE.T`},
\ALG{`SIZE.R.x`}, \ALG{`SIZE.R.y`}, \ALG{`SIZE.R`}, \ALG{`SIZE.*`}.

\CLIConfigLevelII{nodes}{beam}[1]
The number of beams.
Dimensions similar to \hyperref[d:subsubsec:sim_config_nodes_size]{\texttt{size}}.
Its corresponding \hyperref[d:sec:macro]{macro}s in \hyperref[d:sec:calc]{CALC} are
\ALG{`BEAM.T.x`}, \ALG{`BEAM.T.y`}, \ALG{`BEAM.T`},
\ALG{`BEAM.R.x`}, \ALG{`BEAM.R.y`}, \ALG{`BEAM.R`}, \ALG{`BEAM.*`}.

\CLIConfigLevelII{nodes}{grid}[1][same]
The number of grids.
Dimensions similar to \hyperref[d:subsubsec:sim_config_nodes_size]{\texttt{size}}.
The default value is \texttt{same}, which denotes the number of beams is the same as the number of antennas
(or the number of RIS reflecting elements).

Its corresponding \hyperref[d:sec:macro]{macro}s in \hyperref[d:sec:calc]{CALC} are
\ALG{`GRID.T.x`}, \ALG{`GRID.T.y`}, \ALG{`GRID.T`},
\ALG{`GRID.R.x`}, \ALG{`GRID.R.y`}, \ALG{`GRID.R`}, \ALG{`GRID.*`}.

\CLIConfigLevelII{nodes}{beamforming}[1]
For a \hyperref[d:subsec:sim_config_nodes]{node} with \hyperref[d:subsubsec:sim_config_nodes_role]{role}
\texttt{transmitter} (Tx) or \texttt{receiver} (Rx), it is the active beamforming as precoding and combing, respectively.
For a \texttt{RIS} node, it is the \textbf{passive} reflection tensor.
% TODO: The dimension and data type is shown in Table~

\CLIConfigLevelIII{nodes}{beamforming}{variable}[1]
For Tx or Rx, this sets the variable name of the beamforming matrix (for narrowband) or tensor (for wideband).
For RIS, this is the variable name of the reflection tensor.

\CLIConfigLevelIII{nodes}{beamforming}{scheme}[1][random]
Possible value is \texttt{random} (default) and \texttt{custom}.

\CLIConfigLevelIII{nodes}{beamforming}{formula}[2]
Custom beamforming formula in ALG.
This is required if you set \hyperref[d:par:sim_config_nodes_beamforming_scheme]{\texttt{scheme}} to \texttt{custom}.

\CLIConfigLevelI{macro}
User-defined macros.
See \S\ref{d:sec:macro} for details.

\CLIConfigLevelI{channels}[1]
This is a sequence (array) of channel links.
The settings for each channel link is shown below.

\CLIConfigLevelII{channels}{id}[1]
Similar to \hyperref[d:subsubsec:sim_config_nodes_id]{\texttt{id}} in \hyperref[d:subsec:sim_config_nodes]{nodes},
each channel link has a unique identifier as in field \texttt{id}.

\CLIConfigLevelII{channels}{from}[1]
The node \hyperref[d:subsubsec:sim_config_nodes_id]{\texttt{id}} which the link channel is transmitted from.

\CLIConfigLevelII{channels}{to}[1]
The node \hyperref[d:subsubsec:sim_config_nodes_id]{\texttt{id}} which the link channel transmits to.

\CLIConfigLevelII{channels}{sparsity}[1]
The sparsity (number of clusters) of the channel.

\CLIConfigLevelII{channels}{gains}[1][normal]

\CLIConfigLevelI{sounding}[1]
Information related to the sounding procedure is defined here.
The position of \texttt{sounding} is shown in Fig.~\ref{d:fig:sequence_sim}.

\CLIConfigLevelII{sounding}{variables}[1]
Channel variable names are defined here, including
\begin{itemize}[itemsep=0em]
  \item \texttt{received}\indexTTtt{sounding}{variables}{received}: received signal
  \item \texttt{noise}\indexTTtt{sounding}{variables}{noise}: received noise
  \item \texttt{channel}\indexTTtt{sounding}{variables}{channel}: cascaded channel
\end{itemize}

\CLIConfigLevelI{preamble}
This is the code part before main simulation
(including sounding, estimation and report generation).
The position of \texttt{preamble} is shown in Fig.~\ref{d:fig:sequence_sim}.
\begin{tip}
  Custom functions can be defined here.
\end{tip}
This part is specified using the \hyperref[d:chap:alg_lang]{ALG language}.
\begin{figure}[htbp]
  \centering
  \begin{tikzpicture}[
    , n/.style = {
      , draw
      , fill = #1!20
      , minimum width = 64mm
      , minimum height = 6.5mm
    }
    , nt/.style = {
      , darkgray
      , font = \footnotesize\sffamily
      , align = left
    }
    , arr/.style = {
      , -latex
    }
    , thick
    , font = \sffamily
    , node distance = 4mm and 4mm
  ]
    \node (alg-1) [n = myblued, minimum width = 25mm] {Algorithm \#1};
    \node (alg-n) [n = myblued, minimum width = 25mm, right = 10mm of alg-1] {Algorithm \#\textit{n}};
    \foreach \x in {1, n} {
      \node (con-\x) [below = of alg-\x, n = mygreen, minimum width = 25mm] {Conclusion};
      \draw [arr] (alg-\x) -- (con-\x);
    }
    \foreach \x in {alg, con} {
      \node [right = 2mm of \x-1] {$\bm{\cdots}$};
    }
    \begin{scope}[on background layer]
      \node (est) [n = gray, fit = {([yshift = 6mm]alg-1.north west)(con-n)}, thick] {};
    \end{scope}
    \node (est-text) [anchor = north, inner sep = 1.2mm, minimum width = 64mm] at (est.north) {Estimation};
    \node (app) [below = of est, n = mypink] {Appendix};
    \node (sdn) [above = of est, n = myyellow] {Sounding};
    \node (pre) [above = of sdn, n = mypurple] {Preamble};
    \node (std) [above = of pre, n = myred] {Standard ALG};
    \node (set) [above = of std, n = myblues] {\textit{Initial Settings}};
    \draw [arr] (set) -- (std);
    \draw [arr] (std) -- (pre);
    \draw [arr] (pre) -- (sdn);
    \draw [arr] (sdn) -- (est);
    \draw [arr] (est) -- (app);
    \node [nt, right = of set] {This is automatically done by the mmCEsim compiler.};
    \node [nt, right = of std] {Load algorithm implementations set in \S\ref{d:par:sim_config_simulation_jobs_algorithms}.};
    \node [nt, right = of pre] {See \S\ref{d:subsec:sim_config_preamble}.};
    \node [nt, right = of sdn] {See \S\ref{d:subsec:sim_config_sounding}.};
    \node [nt, right = of est-text] {See \S\ref{d:subsec:sim_config_estimation}.};
    \node [nt, right = 6mm-\pgflinewidth of alg-n] {See \S\ref{d:par:sim_config_simulation_jobs_algorithms}.};
    \node [nt, right = 6mm-\pgflinewidth of con-n] {See \S\ref{d:subsec:sim_config_conclusion}.};
    \node [nt, right = of app] {See \S\ref{d:subsec:sim_config_appendix}.};
  \end{tikzpicture}%
  \caption{Sequence of simulation.}
  \label{d:fig:sequence_sim}
\end{figure}

\CLIConfigLevelI{estimation}[1][auto]
This part is specified using the \hyperref[d:chap:alg_lang]{ALG language}.
The position of \texttt{estimation} is shown in Fig.~\ref{d:fig:sequence_sim}.

\CLIConfigLevelI{conclusion}
This part is specified using the \hyperref[d:chap:alg_lang]{ALG language}.
The position of \texttt{conclusion} is shown in Fig.~\ref{d:fig:sequence_sim}.

\CLIConfigLevelI{appendix}
This is the code part after all jobs are done.
The position of \texttt{appendix} is shown in Fig.~\ref{d:fig:sequence_sim}.
This part is specified using the \hyperref[d:chap:alg_lang]{ALG language}.

\CLIConfigLevelI{simulation}[1]
The main simulation configurations.

\CLIConfigLevelII{simulation}{backend}[1][cpp]
Possible values are
\texttt{cpp} (C++ with Armadillo library),
\texttt{python} (Python with NumPy library),
\texttt{matlab}, \texttt{octave}.
This sets the language it exports to and the backend simulation bases on.

\CLIConfigLevelII{simulation}{jobs}[1]
Simulation jobs.
Each job is independent of one another.

\CLIConfigLevelIII{simulation}{jobs}{name}[1]
Simulation job name which is used in the generated report.

\CLIConfigLevelIII[simulation][jobs][test_num]{simulation}{jobs}{test\_num}[1][50]
Number of Monte-Carlo\index{Monte-Carlo} simulations.

\CLIConfigLevelIII{simulation}{jobs}{SNR}[1]
Signal-to-noise ratio\index{signal-to-noise ratio} (SNR\index{SNR}).

\CLIConfigLevelIII[simulation][jobs][SNR_mode]{simulation}{jobs}{SNR\_mode}[1][dB]
The mode of SNR. Possible values are \texttt{dB} (default) and \texttt{linear}.

\CLIConfigLevelIII{simulation}{jobs}{pilot}[1]
The number of pilot overheads.

\CLIConfigLevelIII{simulation}{jobs}{algorithms}[1]

\CLIConfigLevelII{simulation}{report}
Currently, this field is ignored.
Both plain text report (\texttt{.rpt})
and \LaTeX{} report will be generated.

\section{Algorithm}

Algorithm are defined in ALG language,
please refer to \S\ref{d:chap:alg_lang} for details.

\section{Tools}

\subsection{Compose}\index{compose}\indextt{mmcesim-compose}

Compose \texttt{.sim} configuration file from command line options.
\begin{tip}[Note]
  This is still under development
\end{tip}

\subsection{Log}\index{log}\indextt{mmcesim-log}
View or copy mmCEsim log file with \texttt{mmcesim-log}\indextt{mmcesim-log}.

\subsubsection{Log File}
The log file is at \texttt{\textit{<install\_prefix>}/bin/mmcesim.log}.
It stores information about the configuration and many internal processing details.
This is especially useful for diagnosis.

Here is an example of the header part of a log file:
\begin{lstlisting}
[INFO] * Time     : 2023-05-06 10:40:17 (UTC +0800)
[INFO] * Version  : 0.2.1
[INFO] * System   : MacOs-ARM
[INFO] * Compiler : clang v13.0.0
[INFO] * CLI Args : mmcesim "-h"
\end{lstlisting}

\subsubsection{Usage}
The available command line options are listed as follows:
\begin{itemize}
  \item
    \texttt{-v}\indexTtt{mmcesim-log}{-v}
    (\texttt{-{}-version}\indexTtt{mmcesim-log}{-{}-version}):
    produce help message;
  \item
    \texttt{-h}\indexTtt{mmcesim-log}{-h}
    (\texttt{-{}-help}\indexTtt{mmcesim-log}{-{}-help}):
    print version string;
  \item
    \texttt{-p}\indexTtt{mmcesim-log}{-p}
    (\texttt{-{}-print}\indexTtt{mmcesim-log}{-{}-print}):
    print mmCEsim log on terminal;
  \item
    \texttt{-c}\indexTtt{mmcesim-log}{-c}
    (\texttt{-{}-copy}\indexTtt{mmcesim-log}{-{}-copy}):
    copy mmCEsim log to clipboard;
  \item
    \texttt{-f}\indexTtt{mmcesim-log}{-f}
    (\texttt{-{}-file}\indexTtt{mmcesim-log}{-{}-file}):
    show mmCEsim log file location.
\end{itemize}

If no option is provided, it will use the \texttt{-p} option to print the log.
You can use several options together, such as \texttt{-cp} to print and copy.

\begin{tip}
  Use can use \texttt{grep} to filter the log, for example use
  \begin{lstlisting}[language=sh]
mmcesim-log | grep "\[ERROR\]"
  \end{lstlisting}
  to get all error messages.
\end{tip}

\subsection{Maintain}\index{maintain}\indextt{mmcesim-maintain}
You can view the latest stable version available with
\texttt{mmcesim-maintain~-l}\indexTtt{mmcesim-maintain}{-l}.
This internally invokes \texttt{curl https://mmcesim.org/VERSION},
so you need to have \texttt{curl} installed.
