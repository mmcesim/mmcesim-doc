\section{Download Binary}
You can download the built binary of mmCEsim from \href{https://github.com/mmcesim/mmcesim/releases}{GitHub releases}.
The built CLI binaries include support for Linux (x86), macOS (x64 and arm) and Windows (x86).

They all statically link to libraries, so theoretically no dependency is needed.

\begin{tip}[Note]
  Since GitHub Actions currently only provide x86\_64 machines,
  the binary for macOS with arm architecture is built manually on my MacBook Air with an M1 chip.
\end{tip}

\section{Build from Source}

Since mmCEsim is built with CMake, so you can easily build the source on Unix-based systems.
For Windows, I think there are similar ways.

On a Unix-based system, you can simply use the following code to build and install mmCEsim.
\begin{lstlisting}[language=sh, morekeywords={git, cmake, make, sudo}, alsoletter={.}]
git clone https://github.com/mmcesim/mmcesim.git --recurse-submodules
cd mmcesim
cmake . build
cd build
make
sudo make install
\end{lstlisting}

\begin{remark}
  The option \texttt{--recurse-submodules} is required because some dependencies of mmCEsim
  are managed by Git submodules.
\end{remark}

You need to have a C++ compiler that supports C++17 standard,
and have installed the Boost library (statically) of minimum version \texttt{1.70.0} on your system.
You can install them easily on Unix-based systems with your favourite package manager.
For Windows users, please follow the official instruction of Boost.
\begin{lstlisting}[language=sh, morekeywords={sudo, apt, pacman, port, brew}]
# Debian, Ubuntu
sudo apt install libboost-dev
# Arch
sudo pacman -Ss boost
# macOS
sudo port install boost # with MacPorts
brew install boost      # with HomeBrew
\end{lstlisting}

If you want to build the GUI app as well, you need to install Qt6.

\newpage
Some options can be configured when calling \texttt{cmake}.
\begin{itemize}
  \item \textcolor{forestgreen}{\texttt{CMAKE\_BUILD\_TYPE}}: Build type (default as \texttt{Release})
  \item \textcolor{forestgreen}{\texttt{CMAKE\_INSTALL\_PREFIX}}: Installation prefix (default as system path)
  \item \textcolor{forestgreen}{\texttt{MMCESIM\_BUILD\_ASTYLE}}: Build \texttt{astyle} Code Formatter (default as \texttt{ON})
  \item \textcolor{forestgreen}{\texttt{MMCESIM\_BUILD\_GUI}}: Build mmCEsim GUI App with Qt (default as \texttt{OFF})
  \item \textcolor{forestgreen}{\texttt{MMCESIM\_APPLE\_COPY\_SH}}: Copy additional shell script for macOS (default as \texttt{OFF}).
\end{itemize}

For example, you may use \lstinline[morekeywords={cmake}]{cmake . build -D CMAKE_INSTALL_PREFIX=usr/mmcesim}
to install mmCEsim to the directory \texttt{usr/mmcesim}.

\section{Troubleshooting}
\subsection{macOS Safety Warning}
You may view a safety warning after downloading the binary from GitHub Releases.
The \texttt{trust\_mmcesim.sh} is a script to remove that warning.
(Give the script proper permission before running in its directory).
Technically, it does \lstinline[morekeywords={xattr}]{xattr -r -d com.apple.quarantine <binary>}.
