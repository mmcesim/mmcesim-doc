Before diving into documentation details, let's first have a preview of mmCEsim.
Maybe you are not sure whether your research or study need this powerful tool,
then read this chapter to have a glimpse of mmCEsim.

\section{Introduction}

The application is dedicated to simulate
millimeter wave (mmWave) channel estimation:
\[
  \text{mmCEsim}
  =\text{\textbf{mm}Wave}
  +\text{\textbf{C}hannel \textbf{E}stimation}
  +\text{\textbf{sim}ulation},
\]
where reconfigurable intelligent surface (RIS),
also known as intelligent reflecting surface (IRS) \cite{wu2019towards} is supported
for multiple input multiple output (MIMO) systems.

\begin{figure}[htbp]
  \includegraphics[width=\linewidth]{img/badge/mmCEsim_badge.png}
  \caption{mmCEsim banner.}
\end{figure}

We offer a task-oriented simulation software for researchers to focus on algorithms only
without being bothered by coding.

\section{Features}

Here is a list of basic features of mmCEsim:
\begin{itemize}
  \item Task-oriented mmWave channel estimation formulation;
  \item Customizable system model;
  \item Extendable algorithms with our designed ALG language;
  \item Multiple RISs support;
  \item Automatic report generation (in plain text and \LaTeX{} PDF);
  \item Well-written documentation with examples and tutorials.
\end{itemize}

\section{Algorithm Background}

The task-oriented channel estimation for (RIS-assisted) mmWave MIMO systems
is implemented with compressed sensing (CS),
which exploits the sparsity of mmWave channels.

\section{Software Implementation}
Based on the algorithm background,
we implement this software with
command line interface (CLI),
graphic user interface (GUI),
web application and a VS Code extension.
The workflow\index{workflow} of mmCEsim is depicted in Fig.~\ref{fig:mmCEsim_workflow}.

\begin{figure}[htbp]
  \centering\sffamily
  \begin{tikzpicture}[
    , thick
    , node distance = 5mm and 8mm
    , n/.style = {
      , draw
      , fill = #1!20
      , minimum height = 8mm
      , minimum width = 22mm
      , align = center
    }
    , null n/.style = {
      , inner sep = 0
      , outer sep = 0
    }
  ]
    \node (gui) [n = mypurple] {GUI App};
    \node (config) [below = 8mm of gui, n = myblued, font = \small] {Configuration\\File (YAML)};
    \node (alg) [below = 4mm of config, n = myblued, font = \small] {Algorithm\\Library (ALG)};
    \node (export) [right, xshift = 10mm, n = mygreen, font = \large] at ($(config.south east)!.5!(alg.north east)$) {Code\\Exporter};
    \node (py) [n = myyellow, right = 9mm of export, minimum height = 5mm, minimum width = 25mm, font = \small] {Python (NumPy)};
    \node (cpp) [n = myyellow, above = 1mm of py, minimum height = 5mm, minimum width = 25mm, font = \small] {C++ (Armadillo)};
    \node (m) [n = myyellow, below = 1mm of py, minimum height = 5mm, minimum width = 25mm, font = \small] {\textsc{Matlab}/Octave};
    \node (code) [above = .5mm of cpp] {Exported Code};
    \node (run) [n = mygreen, right = 9mm of py, font = \large] {Code\\Runner};
    \node (rpt) [n = mygreen, right = of run, font = \large] {Report\\Generator};
    \draw [-Latex] (run) -- (rpt);
    \draw [-Latex, gray] (gui) -- (config);
    \foreach \x in {config, alg} {
      \draw (\x.east) -| ([xshift = -5mm]export.west);
    }
    \draw [Latex-] (export.west) -- ++ (-5mm, 0);
    \draw [-Latex] (export.east) -- ++ (8mm, 0);
    \draw [Latex-] (run.west) -- ++ (-8mm, 0);
    \node [fit = (code)(m), inner sep = 1mm, draw, dashed, Blue] {};
    \draw [-Latex, gray, dashed] (gui) -| node [null n] (tmp-1) {} (export);
    \draw [-Latex, gray, dashed] (tmp-1) -| node [null n] (tmp-2) {} (run);
    \draw [-Latex, gray, dashed] (tmp-2) -| (rpt);
    % \begin{scope}[on background layer]
    % \end{scope}
  \end{tikzpicture}%
  \caption{mmCEsim workflow.}
  \label{fig:mmCEsim_workflow}
\end{figure}
